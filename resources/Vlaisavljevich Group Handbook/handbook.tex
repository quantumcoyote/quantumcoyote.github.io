\documentclass[letterpaper]{article}

\usepackage{hyperref}
\usepackage{geometry}
\usepackage{fancyhdr}
\usepackage{graphicx}
\usepackage{wrapfig}

% Font Package
\usepackage[T1]{fontenc}
\usepackage{helvet}
\usepackage[usenames,dvipsnames,svgnames,table]{xcolor}

% Set your name here
\definecolor{usd_red}{RGB}{186,12,47}
\def\name{\color{usd_red}{Bess Vlaisavljevich}}

\geometry{
  body={6.5in, 8.5in},
  left=1.0in,
  top=1.25in
}

\usepackage{sectsty}
\usepackage{float}
\sectionfont{\scshape\large}
\subsectionfont{\scshape}

%bibliographay style
\usepackage[super,numbers,sort&compress]{natbib}
\bibliographystyle{InorgChem}
\setlength{\bibsep}{0pt plus 0.3ex}


% Don't indent paragraphs.
\setlength\parindent{0em}

\begin{document}
\pagenumbering{gobble}

\leftline{\Large{\bf{{\color{usd_red}{Vlaisavljevich Group Handbook}}-- Updated August 2021}}}
\noindent\rule{16.5cm}{1.5pt}

\section{Purpose}
Starting a research project is full of challenges but knowing what is expected of you and where to go for help should not be one of them. This handbook is an evolving document about Prof. Vlaisavljevich's philosophy about how the group should function. The group members views are very important as well. So, if you have suggestions, share them!

\section{Welcome}
Welcome to the computational chemistry team at USD!  Prof. Vlaisavljevich has invited you to join her group because she is very excited about working hard with you on difficult and interesting problems.\\

As a research group, our goal (which we can achieve consistently) is to perform world-class research. All of the members of the group have made a decision to make our research one of the top priorities in their lives. This commitment makes the group very successful. It also makes being a group member a (more than) full time job. The work you put in and the results you get out are coupled. The time you spend in the computational lab will vary at different points in your career and project, but a willingness to be fully committed to your problem is a very important starting point.\\

I also want the computational lab to be a fun and rewarding environment for all of us, and we need to work together to provide deep and meaningful insights into the properties of challenging molecules and materials. The goal of this document is to make it clear what is expected of all of us, so that we can have fun while ensuring that our work is of the highest quality. It outlines what you can expect from your advisor and what the members of the computational lab can expect from each other.\\

If you ever feel that a member of the group (including the PI) is not meeting the expectations in this document please let Prof. Vlaisavljevich know. This is especially important if you feel unwelcome in the group, are otherwise unable to perform to best of your ability, or suspect potential academic dishonesty. All the members of the computational lab must be committed to fixing problems that arise. If you have a problem that you do not feel comfortable talking to Dr. Vlaisavljevich about it, Prof. Jiang, Prof. Miro, and Prof. Sykes are all good resources and can help moderate a conversation. If you need assistance with the University (things outside of Prof. Vlaisavljevich's control) you can always start by contacting the Dean of Students Office. They are fantastic at helping you find the correct resources. The only exception is if you are a postdoc, in which case you will likely need to go directly to HR. \\

This document is not going to cover every possible situation, but I hope it covers the big picture and most likely scenarios. I will add to it when I realize there is something missing or if group policy changes. I expect you to read all of it.\\

This document is primarily aimed at post docs and graduate students. Undergraduate students should read through the document and pay attention to the part specifically for you (the general expectations apply to you but you should use common sense to figure out which ones are for full time researchers and which still apply to you). Ask if you are not sure how this document applies to you.\\

It should go without saying, but anything in this document conflicts with your employment contract with the university or university rules, the contract or rules takes precedence. (Please let Dr. Vlaisavljevich know if you think this is the case, so she can resolve the conflict and write this document with additional clarity).\\

Once again, welcome to the computational team. I am very excited to have you here!

\section{Expectations}
\subsection{For Everyone}
\begin{enumerate}
\item Treat everyone in the group with respect. You do not need to be friends with everyone (although it is nice when it works out that way), but the work environment should be pleasant and professional. 
\item Work as part of the team. There is  no tolerance for any attempt to undermine another lab member's research or academic efforts.
\item Behave honestly and, in particular, do not plagiarize or fabricate data. Science only works when we all do this. If you have ever tried to reproduce someone else's work from the literature, you likely found this challenging even when the researcher made every effort to give all details regarding their work. When data is fabricated this is a HUGE waste of everyone's time and scientific progress is held back. Sadly, it does happen even though it should not!
\item Work hard to support the group's goal of doing world-class research. Research should be a top priority for you. It is why we are all here. 
\item All research communication within the group will be through Microsoft Teams and not by email.
\item Be available during the work week. This includes checking your email and Teams and replying promptly. We cannot learn from each other if we do not talk. See section below about work hours for more details.
\item Attend all departmental seminars, including the student seminar series. It is very important for scientists to hear about research in and outside of their field. The Chemistry Department's seminars are a great opportunity for you to expand your horizons. 
\item Attend all master's and Ph.D. thesis defenses within the chemistry department.
\item You are also encouraged to volunteer to go to lunch/dinner with the speaker and attend the student meetings with speakers.
\item Follow through on the commitments you make.
\item Show up to meetings on time. On the rare occasion when something comes up and you are running late, let someone know.
\item Meet deadlines. This includes coming to weekly meetings prepared and 
\item If you are sick or not coming to work, let someone know. If you have not seen a group member for a long time (especially if their advisor is traveling or unaware), let Dr. Vlaisavljevich know. We have to take care of each other.
\item Take care of your mental heath and keep and eye out for your co-workers (both in our group and other groups). 1 in 3 graduate students seek treatment for mental health during their degrees. Here is a link to a well-known computational chemistry professor talking about his experience and I think it is worth a read. (http://pollux.chem.umn.edu/MentalHealth.html)
\item Keep the lab clean (easier for us than in experimental groups but still applies) and in good repair. Make sure everything is working as it should.
\item Provide thoughtful, constructive feedback on other lab members' work.
\item Be a good ambassador for our lab. Represent us professionally. We want to recruit new members and get other groups excited about our work. This benefits all of us and we can only do this together.
\item Do not share unpublished data without your advisor's permission except with other members of the computational groups at USD. See the section on collaborations for more info. This does not mean you cannot talk about your project with your friends (that is highly encouraged!). This means prematurely sharing hard data and reports outside the group.
\item Work together to resolve any conflicts that arise in the group. Dr. Vlaisavljevich or Dr. Miro are happy to mediate when necessary.
\item Keep up with the literature in your area. I try my best to do the same but you should not rely on your advisor to find all the papers you should read. In fact, I will rely on you to be sure we are not missing something and by the time we write the paper, you should know more about the study than I do! It is your work! Read, read, and read!
\item At the end of the day, you are responsible for your project and progress. Your advisor and the rest of the team are here to support you but by the end of your time in the group, you should be able to do similar projects from start to finish on your own. The best way to learn this is to take more and more ownership in the projects as your time in the group progresses.
\end{enumerate}

\subsection{Microsoft Teams}
Everyone is a member of the Theory Group and Vlaisavljevich Group Teams. You will also be added to a Teams for your specific research project(s). Please use teams for all communications within the group. If you have a question about a generic code, go to the theory groups teams, and post in that channel. If you have a question about a project, post it in that Teams chat. This is so everyone doing related work can see the question. If you ask me a question about BAGEL in the chat function in Teams, then no one else can see it and learn from the answer. It would be the same as writing an email. If you have a question for another group member, teams is also a great way to ask them.  
\begin{enumerate}
\item Always tag the group or specific person you are addressing when posting. If you are writing only for a specific person, tag that person but don't expect anyone else in the group to see your post in a timely manner. If you don't tag anyone, then your post might not get read.
\item Acknowledge that you have read a post (when appropriate) but liking the post. 
\item All working documents should be in Teams and updated 'live'.  Reports should be must up to date (and 'live') in the teams associated with that project. When we review reports every three weeks, you should copy your report to the Vlaisavljevich Group Teams to the Report channel. When preparing a talk, the slides should also be shared through the respective Project in Teams to get feedback. 
\item If you are analyzing data in a spreadsheet, including the file in Teams is also a good idea. The file is backed up if your computer crashes.  
\end{enumerate}

\subsection{Advisor's responsibilities}

\begin{enumerate}
\item Help you become an excellent researcher.
\item Provide honest, thoughtful feedback on your work.
\item Provide a productive, safe, and fun work environment.
\item Provide you with the resources you need to learn and be successful.
\item Assist you in your professional development and give advice on any professional decisions (when the advice is wanted!).
\item Set aggressive, but reasonable research goals for the group.
\end{enumerate}

If you need to talk to Dr. Vlaisavljevich regarding science or otherwise, please do not hesitate to ask. If I cannot meet with you at that immediate moment, I may ask to schedule a meeting time. Speaking with you is never an imposition, it is the best part of my job! But sometimes I have to schedule around our other commitment (teaching duties in particular). I have shared my calendar with you to help us find a time easily. On a similar note, if you have concerns or complaints, see me too.

\subsection*{Postdoctoral Researchers}
You have earned a Ph.D. degree, which is no easy task. Through this work, you have demonstrated the ability to think creatively and independently. I will not only encourage this but will also rely on it! As such, there are some expectations unique to postdocs. 

\begin{enumerate} 
\item Your position is a 100\% research position. Take the lead, be excited, explore new science in the group.
\item Make suggestions! Be creative! Be a leader!
\item If you want to start a new collaboration, a new project, or a totally new line of research, that is great. But be sure to talk to Dr. Vlaisavljevich first. Generally I am supportive of this but the funding that supports you is connected to a particular project and we have to make sure that the way you are spending your time makes sense.
\item Be willing and excited to present your work at national meetings. It is your job to suggest conferences you want to attend. Let your advisor know what your interest are and we can talk about the right number/type of meetings to attend each year (and the funds available to make this happen).
\item Support the students. Be a resource for them, answer their questions, help them when you can.
\item Write the first draft of manuscripts.
\item Lead your collaborations with experimental groups.
\item Start developing independent research ideas. What might you propose to do? What interests you outside of the group's research lines? How would you start writing a proposal? The postdoc is a great time to think about these things and it can help you in deciding what the next step in your career will be. 
\item You are welcome to assist me with teaching, but only if you feel this will help your career progress. It is not required.
\end{enumerate}

\subsection*{Graduate Students}
Graduate students span a huge range of experiences and abilities. A first year graduate student should, for obvious reasons, be held to different standards than a fourth year graduate student.\\

In the first few years, coursework, literature familiarization, software familiarization, and obtaining a deep understanding of the methods and chemistry you are exploring will occupy most of your time. Later, you will become much more efficient in the research and can focus your efforts on improving how you communicate your results (writing papers, giving seminars, etc). \\

That said, by the second spring (at the latest) you should have made significant progress on your research. You all have the deadline of writing your M.S. thesis, and for PhD students your project proposal will come faster than you think. M.S. students considering continuing for a Ph.D. in the group must contact the advisor at least 6 months in advance so his/her request can be evaluated and planned in a timely manner. You will need to reapply to the graduate school at USD and funding is not always easy, so the sooner you share your plans, the better. Even if Dr. Vlaisavljevich wants to keep you in the group, the department committee will evaluate your application and compare to all the other applications that year. Some years, there are more PhD spots open than others so the sooner you know, the more opportunities you have to apply.. \\

Graduate school is about learning but it is perhaps most about time management.  There are key points at which you will be evaluated and you have a limited time to prepare for them. While it might be interesting to learn about EVERYTHING in chemistry, there are areas that will be more important for you to focus your efforts on, and those areas and the type of effort will need to change as you progress in your degree. The group is here to help with this, but as you progress in the program, only you will be able to really identify what the best approach is. It is your degree! Make it your own!

\begin{enumerate}
\item You must pass the four ACS exams.
\item You must pass the graduate coursework. Often graduate students say grades are not important. This attitude drives Dr. Vlaisavljevich particularly crazy since mastering the material is very important. Grades are only one measure of this (that is true) but I expect you to pursue excellence in all of your work (and ideally getting good grades won't mean that research progress suffers). Additionally, your stipend and status in the program depends on maintaining a certain GPA.
\item Ph.D students must pass four cumulative exams. 
\item You should take all the freedom you can handle in your research. In fact, I will try to give you a bit more freedom than you might think I should.  I will do this faster for you than for the undergraduate researchers. If you think I am striking the wrong balance, let me know.
\item By the end of your degree (sooner if  you can!), you are expected to take on the same role in the group as the postdocs. The only difference is that you will need to write a thesis and because of this your research projects will all fall under a focused encompassing theme.
\item Present your work at a conference a minimum of one time per year (posters are ok at the beginning but you should progress to oral presentations at conferences by the midpoint of your degree). This is in addition to the ones required for your M.S. or Ph.D. degree.
\item Help the undergraduates, help your peers, and also help the postdocs (they will have questions for you too!)
\item Group meeting presentations should be as close to public presentations as possible. The story may not have an ending yet, but the introduction and results sections should be presented in a professional way. See section on presentations below.
\item Try to solve your questions by yourself before asking other people. A huge part of graduate school is learning how to teach yourself.
\item You have a limited number of days during your degree and it is a lot of work. A good goal is to feel like you are working harder than everyone else in the group.
\end{enumerate}


\subsection*{Undergraduate Students}

\begin{enumerate}
\item Ask questions.
\item Make suggestions for new directions if you see them.
\item Take advantage of the postdocs and graduate students in the group. They are there to help you! 
\item Treat your project as a learning experience.
\end{enumerate}

If receiving credit for research, you must commit to performing 3 hours of research per week for each credit earned. If the position is paid, it goes without saying that you should work the hours you are paid to work. Research on a voluntary basis can also be performed. We will agree to the amount of time you want to commit, since the research project should be chosen in a way that it can be accomplished within the time constraints of your schedule. If you need to change your working hours, see me and we will find a solution.

\section*{Presentations}
Presenting your work in writing, giving talks, and posters are the primary mechanisms we use to disseminate our results and influence our field. Since we aim to do world-class research at USD, we must also strive to produce world-class publications and presentations. In most cases, new graduate students have not had much experience with either preparing publication-worthy research reports or giving presentations that rise to that standard. That is alright, since you will learn this during your time in graduate school. This takes time and effort and something scientists work on for their whole career. When you have a presentation or written deadline, you will have to put in a lot of hours leading up to it. This will mean putting in longer hours than you may normally do (and while preparing a talk takes a long time but it should not mean your research stops completely).\\

Posters and presentations should be approved by your supervisor since you are going out to represent the group and USD. It is your responsibility to ensure that all the data and references in the presentation are correct. Your supervisor will provide advice about how to convey your story and how to make your topic easy to understand for the audience.

\begin{enumerate}
\item Finished presentations must be shown to the advisor at least one week before presenting. If this is not done, you will withdraw your talk.
\item No unpublished data should be shown publicly without permission from your advisor.
\end{enumerate}


\section*{Safety}
Unlike experimental labs, our safety concerns are similar to any office worker. If you want someone to check if your office space is set up in the best way, let us know. If you need a new keyboard, chair or whatever, let us know and we will do our best to solve the problem. There are foot rests in boxes in UPEL 201. Please use them if they help you.

\section*{Reports}
Communicating your work in writing is among the most important skill you will learn. This includes having a clear flow to the paper (a story) and having professional figures/tables/graphs. To help you improve this skill and assist in developing better time management skills, a report for each of your projects should be updated every 3 weeks, at a minimum. Dr. Vlaisavljevich will not micromanage this but asks that you take responsibility in keeping your report file updated. She will periodically open the reports in Teams when she needs to share your results in her talks or grant reporting. See Monday meetings for more information about reviewing reports.

At the start, your report will only include an introduction to the project and computational details. As you add results, you will need to revisit the organization of the report.  The report is not a publication. You should not try to write a full introduction before you know what your results say. Think of your report as the results section of your manuscript. The idea is that when the project is done, your results and discussion section will be written! Then you will only need to write the abstract, introduction, and conclusions to tell the story of your publication. You can also include Supporting Information at the end of your report so that compiling this information is easier when it's time to publish.

You can add references with EndNote as you go or add them as comments in Word. 

I would also recommend using comments to solicit feedback. For example, if you are unsure of your interpretation, you can note that with a comment. Or you may flag a particular sentence where you aren't satisfied with the wording.

\section*{Time in the Lab}
A research lab should not be a place where you are forced to appear at 9 am and wait for the clock to count down at the end of the day. However, I am usually on campus between 8:30 AM and 5:00 PM. When necessary, the work day gets longer along with the occasional weekend and holiday. When not in the office, I am always "logged in" between 7 am and 10 pm. If you are spending a similar amount of effort, great! Of course there is not a one-size-fits-all standard. That said, postdocs are expected to know what effort is needed to produce research (if we are not on the same page, I will let you know). A PhD can go slow or quickly based on your commitment (and some luck too, good or bad). I care more about the results and that you are understanding your work than whether you are in your office every time I walk by. However, as noted above, I have always found discussing with my colleagues to be an important part of research. For this reason, I expect all members of the group to all be in the office roughly between 10AM and 4PM (whether you are in the group that arrives early in the day or the one who prefers to arrive later is your business but being physically present in on campus 8 hours a day, on average, is expected). Remote work should be discussed and approved by Dr. Vlaisavljevich.

You MUST put in the time to achieve enough research for your M.S. or Ph.D. Remember, it is YOUR degree. If you have concerns due to care giving responsibilities, health, etc. let me know so that we can figure out a plan that will not impact your progress.

\section*{Vacations}
For postdocs, vacation time is defined by your contract. Follow the rules of HR in reporting vacation and let me know in advance of your plans. Graduate students are slightly different since you have a specific goal and in the first few years in particular, you face a huge learning curve. The key thing is to not burn yourself out, but less time working is also less progress towards that goal. Part of the Ph.D. is learning how to manage your work commitments and maintain a healthy life outside of work. You also will need to put a different amount of effort into your learning at different stages in your degree. If you have a vacation planned, let me know in advance. As a rough guide, graduate students should expect to have around 2 weeks of vacation per year outside of university holidays (the same amount staff at USD are allocated). If you want to travel longer than two weeks especially internationally, discuss this with your advisor prior to purchasing tickets since we need to be sure that your degree progress will not be hindered. Graduate students should also let Ruth know if they are traveling for an extended period of time.

\section*{Graduate Student Progress Reports}
At the end of each semester, graduate students are required to write a progress report for the Chemistry Department. This is very important and should be taken seriously. The more your report takes the form of a manuscript in progress, the more useful it will be for you when it is time to write the actual paper and/or your thesis. These reports should include, at a minimum, the motivation, results, and future directions of your current project so that any chemistry faculty member can understand the project. If you are an M.S. student and want to apply for the Ph.D. program at USD, the same committee who determines which candidates are accepted, reviews the progress reports. The section you write, your progress towards your MS, and the comments the advisor writes about your progress can all help you. If the committee is seeing great work, they will have no reason not to admit you to the PhD program (if we have the money).

\section*{Monday Meetings}
Monday meetings have three categories: 1) Science Communication and Career Development, 2) Diversity and Inclusive Excellence, and 3) Group Research Update. For 1) the postdocs or Dr. Vlaisavljevich will lead a discussion on a topic chosen the week before. They will assign a task to the group to complete prior to the meeting. For 2), we will read a recent (published in the last 5 years) peer-reviewed paper on a diversity topic in science. This is so we can learn and be better citizens in the chemistry community. For 3), everyone will read a report of a colleague and provide meaningful feedback. One of your reports will also be read. In the following Monday meeting, Dr. Vlaisavljevich will lead a conversation about research progress. You should always have done something to prepare for the Monday meeting. If you are assigned to choose a topic, you must notify the group no later than the end of the workday on Thursday (the week prior) so people can prepare on Friday if they prefer not to work over the weekend. If you don't know on Friday what to do for Monday, ask! 

\section*{Group Meetings}
Weekly group meetings are scheduled each semester based on the group member's courses and teaching responsibilities. They shared with the Mir\'o group and are the perfect opportunity for students and postdocs to develop their communication skills and get feedback on their research. The meeting format is a formal research talk by one member of the computational groups. While the research may not be complete, the slides should be polished and carefully checked for mistakes prior to presenting. The only difference from a public talk is that you can present a challenge you are facing and show confusing data. You should still consider how you are presenting this (to give the audience context to best help you answer your question). \\

Group meeting presentations are intended to be about research progress; however, on occasion you may be asked to present on a topic (a field of chemistry, a computational method, or an experimental technique) either because this topic is essential in your research or because it is of general interest that the group learns more about it. \\

All of the non-presenting group members should ask questions at group meeting. While it can sometimes feel like you are picking on your lab mate if you ask a tough question, remember that group meeting is the best place to get asked something hard (it is the lowest stakes presentation that you have). The better the presentation is prepared, the more meaningful the audience questions will be. Both Dr. Mir\'o and Dr. Vlaisavljevich (and most likely the postdocs too) have countless examples of things they discovered about a project in group meeting that would have been very embarrassing to find out at a conference.  

\section*{Subgroup Meetings}
Every week, the group will meet in three subgroups (Heavy Element Chemistry Projects, DOE Projects, and Polymer Projects). Some group members will be in one subgroup, while others will be in more than one. This meeting is informal but a chance for you to update us on what you have done that week. It is your chance to talk to us. It does not need to be a formal report, but if you are stuck on a research problem, you should prepare your results in a way that most clearly shows the problem. Do not come empty handed. If you have no questions, your part of the meeting may last 5 minutes (a simple update). If you have a lot of things you want to talk about, we may spend most of the time on your project. If you find that you are dominating the meeting every week with technical questions, consider scheduling individual meetings with Dr. Vlaisavljevich more often.

\subsection*{Collaborations}
Collaborations are a vital part of scientific research and the Vlaisavljevich group works closely with experimentalists. When you agree to engage in a collaborative project, you are committing to providing the collaborator with the highest possible quality of data. The group policy is that with each set of data you provide, you must include the computational details (journal style). While I may get annoyed if you send me a table of data with typos in it, it is very likely that I will catch an obvious error. You cannot expect an experimental collaborator to identify when a particular method fails or when your calculation does not compare well with their measurement. It is the burden of the computational chemist to ensure that the right calculations is performed for comparison with a particular experiment. Additionally, errors in your data could lead an experimental group to perform a costly or time consuming experiment for no reason. When you first start collaborating with experimental groups, any reports or data that is to be sent to a collaborator must be approved by your advisor. I will work with you as you learn best practices in communicating your results. As you gain more experience (or if you are a postdoc), you will become responsible for managing these collaborations yourself and that includes knowing when to ask if you are not sure about a result. If you do not conform to the above criteria, you will be removed from the collaborative project. This will be particularly problematic if your funding is related to this project. If you have a question about when you can freely send data, and when you should get approval, always ask.

%\section*{The Paper Writing Process}

%\section*{Guildelines for Figures and Plots}
%

\section*{Searching and Reading the Literature}
Everyone in the group should be following the ACS journals, RSC journals, APS, AIP, and Wiley journals at a minimum. I use an RSS feed to manage this. If you do not know how to follow the literature, ask! If there is work related to your project, it's your responsibility to know about it.

%\section*{Storing Data and Good Lab Practices}



\end{document}
