\documentclass[letterpaper]{article}

\usepackage{hyperref}
\usepackage{geometry}
\usepackage{fancyhdr}
\usepackage{graphicx}
\usepackage{wrapfig}

% Font Package
\usepackage[T1]{fontenc}
\usepackage{helvet}
\usepackage[usenames,dvipsnames,svgnames,table]{xcolor}

% Set your name here
\definecolor{usd_red}{RGB}{186,12,47}
\def\name{\color{usd_red}{Bess Vlaisavljevich}}

\geometry{
  body={6.5in, 8.5in},
  left=1.0in,
  top=1.25in
}

\usepackage{sectsty}
\usepackage{float}
\sectionfont{\scshape\large}
\subsectionfont{\scshape}

%bibliographay style
\usepackage[super,numbers,sort&compress]{natbib}
\bibliographystyle{InorgChem}
\setlength{\bibsep}{0pt plus 0.3ex}


% Don't indent paragraphs.
\setlength\parindent{0em}

\begin{document}
\pagenumbering{gobble}

\leftline{\Large{\bf{{\color{usd_red}{Miro  Group Handbook}}-- Updated July 2022}}}
\noindent\rule{16.5cm}{1.5pt}

%\section*{\centering{\color{black}{Summary of Research Interests}} }

\section{Purpose}
Starting a research project is full of challenges but knowing what is expected of you and where to go for help should not be one of them. This handbook is an evolving document about the Miro group philosophy about how the group should function, but the group members views are very important to us too. So, if you have suggestions, let me know!

\section{Group Meeting Presentations}
Welcome to the Miro group at USD!  I invited you to join my research group because I'm very excited about working hard with you on difficult and interesting problems.\\

As a research group, our goal (which we can achieve consistently) is to perform world-class research. All of the members of the group have made a decision to make our research one of the top priorities in their lives. This commitment makes the group very successful. It also makes being a group member a full time job. The work you put in and the results you get out are coupled. The time you spend in the computational lab will vary at different points in your career and project, but a willingness to be fully committed to your problem is a very important starting point.\\

I also want the computational lab to be a fun and rewarding environment for all of us, and we need to work together to provide deep and meaningful insights into the properties of challenging molecules and materials. The goal of this document is to make it clear what is expected of all of us, so that we can have fun while ensuring that our work is of the highest quality. It outlines what you can expect from your advisor (me) and what the members of the computational lab can expect from each other.\\

If you ever feel that a member of the group (including myself) is not meeting the expectations in this document please let me know. This is especially important if you feel unwelcome in the group, are otherwise unable to perform to best of your ability, or suspect potential academic dishonesty. All the members of the computational lab must be committed to fixing problems that arise.\\

This document is not going to cover every possible situation, but we hope it covers the big picture and most likely scenarios. We will add to it when we realize there is something missing or if there has been some miscommunication or misunderstanding. We expect you to read all of it.\\

This document is primarily aimed at post docs and graduate students pursuing a Ph.D. or M.S. degree. Undergraduate students should read through the document and pay attention to the part specifically for you (the general expectations apply to you but you should use common sense to figure out which ones are for full time researchers and which still apply to you). Of course ask if you are not sure how this document applies to you.\\

It should go without saying, but if anything in this document conflicts with your employment contract with the university or university rules, the contract or rules takes precedence (Please let me know if you think this is the case, so we can resolve the conflict and write this document with additional clarity).\\

Once again, welcome to the computational team. We are very excited to have you here!

\section{Expectations}
\subsection{For Everyone}
\begin{enumerate}
\item Treat everyone in the group with respect. You do not need to be friends with everyone (although it is nice when it works out that way), but the work environment should be pleasant and professional. 
\item Work as part of the team.  There is  no tolerance for any attempt to undermine another lab member's research or academic efforts.
\item To behave honestly and, in particular, to not plagiarize or fabricate data. Science only works when we all do this. If you have ever tried to reproduce someone else's work from the literature, you likely found this challenging even when the researcher made every effort to give all details regarding their work. When data is fabricated this is a HUGE waste of everyone's time and scientific progress is held back. Sadly, it does happen even though it should not!
\item Work hard to support the group's goal of doing world-class research. Research should be a high priority for you. It is why we are all here. 
\item Be available during the work week. If you need to frequently work from home, you need to discuss it with me but this should only occur under very special circumstances. We cannot learn from each other if we do not talk and being on campus facilitates this. See section below about work hours for more details.
\item Attend all departmental seminars, including the student seminar series. It is very important for scientists to hear about research in and outside of their field. The Chemistry Department's seminars are a great opportunity for you to expand your horizons. 
\item Attend master's and Ph.D. thesis defenses within the chemistry department.
\item You are also encouraged to volunteer to go to lunch with the speaker and attend the student meetings with speakers. 
\item Follow through on the commitments you make.
\item Show up to meetings on time. On the rare occasion when something comes up and you are running late, let someone know.
\item If you are sick or not coming to work, let someone know. If you have not seen a group member for a long time (especially if their advisor is traveling or unaware), let us know. We have to take care of each other.
\item Take care of your mental heath and keep and eye out for your co-workers (both in our group and other groups). 1 in 3 graduate students seek treatment for mental health during their degrees. Here is a link to a well-known computational chemistry professor talking about his experience and I think it is worth a read. (http://pollux.chem.umn.edu/MentalHealth.html)
\item Keep the lab clean (easier for us than in experimental groups but still applies).
\item Provide thoughtful, constructive feedback on other lab members' work.
\item Be a good ambassador for our lab. Represent us professionally. We want to recruit new members and get other groups excited about our work. This benefits all of us and we can only do this together.
\item Do not share unpublished data without my permission except with other members of the computational groups at USD. See the section on collaborations for more info. This does not mean you cannot talk about your project with your friends (that is highly encouraged!). This means prematurely sharing hard data and reports outside the group.
\item Work together to resolve any conflicts that arise in the group. I'm happy to mediate when necessary.
\item Keep up with the literature in your area. We try our best to do the same but you should not rely on your advisor to find all the papers you should read. In fact, we will rely on you to be sure we are not missing something and by the time we write the paper, you will likely know more about the study than we do! It is your work! Read, read, and read!
\end{enumerate}

\subsection{Advisor's responsibilities}

\begin{enumerate}
\item Help you become an excellent researcher.
\item Provide honest, thoughtful feedback on your work.
\item Provide a productive, safe, and fun work environment.
\item Provide you with the resources you need to learn and be successful.
\item Assist you in your professional development and give advice on any professional decisions (when the advice is wanted!).
\item Set aggressive, but reasonable research goals for the group.
\end{enumerate}

If you need to talk to me regarding science or otherwise, please do not hesitate to ask for a meeting. If I cannot meet with you at that immediate moment, I may ask to schedule a meeting time through teams. Speaking with you is never an imposition, it is the best part of our jobs! But sometimes I have to schedule around our other commitment (teaching and service duties in particular).

\subsection*{Postdoctoral Researchers}
You have earned a Ph.D. degree, which is no easy task. Through this work, you have demonstrated the ability to think creatively and independently. We not only encourage this but will also rely on it! Because of this, there are some expectations unique to postdocs. 

\begin{enumerate} 
\item Your position is a 100\% research position. Take the lead, be excited, explore new science in the group.
\item Make suggestions! Be creative! Be a leader!
\item If you want to start a new collaboration, a new project, or a totally new line of research, that is great. But be sure to talk to me first. Generally I'm supportive of this but the funding that supports you is connected to a particular project and I have to make sure that the way you are spending your time makes sense.
\item Be willing and excited to present your work at national meetings. It is your job to suggest conferences you want to attend. Let me know what your interest are and we can talk about the right number/type of meetings to attend each year.
\item Support the students. Be a resource for them, answer their questions, help them when you can.
\item Write the first draft of manuscripts.
\item Lead collaborations with experimental groups.
\item Start developing independent research ideas. What might you propose to do? What interests you outside of the group's research lines? How would you start writing a proposal? The postdoc is a great time to think about these things and it can help you in deciding what the next step in your career will be. 
\end{enumerate}

\subsection*{Graduate Students}
Graduate students span a huge range of experiences and abilities. A first year graduate student should, for obvious reasons, be held to different standards than a fourth year graduate student.\\

In the first few years, coursework, reading literature, software familiarization, and obtaining a deep understanding of the methods and chemistry you are exploring will occupy most of your time. Later, you will become much more efficient in the research and can focus your efforts on improving how you communicate your results (e.g., writing papers, giving seminars, etc). \\

That said, by the second spring (at the latest) you should have made significant progress on your research. For M.S. students, you have the deadline of writing your dissertation, and for PhD students your candidacy exam will be rapidly approaching. M.S. students considering continuing for a Ph.D. in the group must contact the advisor at least 6 months in advance so his/her request can be evaluated and planned in a timely manner. Last minute requests never work well with our graduate program.\\

Graduate school is about learning but it is perhaps most about time management. There are key points at which you will be evaluated and you have a limited time to prepare for them. While it might be interesting to learn about EVERYTHING in chemistry, there are areas that will be more important for you to focus your efforts on, and those areas and the type of effort will need to change as you progress in your degree. The group is here to help with this, but as you go only you will be able to really identify what the best approach is. It is your degree! Make it your own! 

\begin{enumerate}
\item You must pass the four ACS exams.
\item You must pass the graduate coursework. Often graduate students say grades are not important. This is wrong since mastering the material is very important. Grades are only one measure of this (that is true) but we expect you to pursue excellence in all of your work.  Additionally, your stipend and status in the program depends on maintaining a certain GPA.
\item Ph.D students must pass four cumulative exams. 
\item You should take all the freedom you can handle in your research. In fact, I will try to give you a bit more freedom than you might think I should.  I will do this faster for you than for the undergraduate researchers. If you think I striking the wrong balance, let me know.
\item By the end of your degree (sooner if  you can!), you are expected to take on the same role in the group as the postdocs. The only difference is that you will need to write a thesis and because of this your research projects will all fall under an encompassing theme.
\item Present your work at a conference a minimum of one time per year (posters are ok at the beginning but you should progress to oral presentations at conferences by the midpoint of your degree). This is in addition to the ones required for your M.S. or Ph.D. degree. 
\item All M.S. and Ph.D. students must present their work at IdeaFest every year.
\item Help the undergraduates, help your peers, and also help the postdocs (they will have questions for you too!)
\item Group meeting presentations should be as close to public presentations as possible. The story may not have an ending yet, but the introduction and results sections should be presented in a professional way. See section on presentations below.
\item Try to solve your question by yourself before asking other people. A huge part of graduate school is learning how to teach yourself.
\item You should not expect me to know all the answers or the solution to all problems. Research does not work like that.
\item You have a limited number of days during your degree and it is a lot of work. A good goal is to feel like you are working harder than everyone else in the group.
\end{enumerate}


\subsection*{Undergraduate Students}

\begin{enumerate}
\item Ask questions.
\item Make suggestions for new directions if you see them.
\item Take advantage of the postdocs and graduate students in the group. They are there to help you! 
\item Treat your project as a learning experience.
\end{enumerate}

If receiving credit for research, you must commit to performing 3 hours of research per week for each credit earned. If the position is paid, it goes without saying that you should work the hours you are paid to work. Research on a voluntary basis can also be performed. We will agree to the amount of time you want to commit, since the research project should be chosen in a way that it can be accomplished within the time constraints of your schedule. If you need to change your working hours, see me and we will find a solution. We need to agree prior to start paid research, you cannot decide that without my approval.


\section*{Presentations}
Presenting our work in writing, giving talks, and posters are the primary mechanisms we use to disseminate our results and influence our field. Since we aim to do world-class research at USD, we must also strive to produce world-class publications and presentations. In most cases, new graduate students have not had much experience with either preparing publication-worthy research reports or giving presentations that rise to that standard. That is alright, since you will learn this during your time in graduate school. This takes time and effort and something scientists work on for their whole career. When you have a presentation or written deadline, you will have to put in a lot of hours leading up to it. This will mean putting in longer hours than you may normally do (preparing a talk takes a long time but it should not mean your research stops completely).Group meeting presentations should be complete, high quality, and scientifically correct. You are going to present at least once in group meeting every semester (including summer).\\

Posters and presentations should be approved by your supervisor since you are going out to represent the group and USD. It is your responsibility to ensure that all the data and references in the presentation are correct. Your supervisor will provide advice about how to convey your story and how to make your topic easy to understand for the audience.


\section*{Safety}

Unlike experimental labs, our safety concerns are similar to any office worker. If you want someone to check if your office space is set up in the best way, let me know. If you need a new keyboard or chair or whatever, let us know and we will do our best to solve the problem ASAP. For group members shared with experimental groups, you are more than welcome to ask for any safety items you need to me. Safety first. 

\section*{Time in the Lab}
A research lab should not be a place where you are forced to appear at 9 am and wait for the clock to count down at the end of the day. However, I usually on campus between 7:30 AM and 5:00 PM. When necessary, the work day gets longer along with the occasional weekend and holiday. When not in the office, I  "logged in" between 8 am and midnight. If you are spending a similar amount of effort, great! Of course there is not a one-size-fits-all standard. That said, postdocs are expected to know what effort is needed to produce research (if we are not on the same page, we will let you know). A PhD can go slow or quickly based on your commitment (and some luck too, good or bad). I have always found discussing with my colleagues to be an important part of research. For this reason, we expect all the group members to be in the office roughly between 9AM and 5PM since being physically present in on campus 8 hours a day is expected. You MUST put in the time to achieve enough research for your M.S. or Ph.D. Remember, it is YOUR degree. I cannot and will not do your research for you. If you have special needs because of children, visitors, etc. just let me know so that we can figure out a plan that will not impact your progress. \textbf{If you systematically do not work near the 40h a week without a valid reason, you will be asked to seek for another research group to complete your degree.}

\section*{Vacations}
For postdocs, vacation time is defined by your contract. Follow the rules of HR in reporting vacation and let us know in advance of your plans. \textbf{You must take your vacation since I do not have the funds to pay overtime.} Graduate students are slightly different since you have a specific goal and in the first few years in particular, you face a huge learning curve. The key thing is to not burn yourself out, but less time working is also less progress towards that goal. Part of the Ph.D. is learning how to manage your work commitments and maintain a healthy life outside of work. If you have a vacation planned, let me know in advance. \textbf{If you find yourself telling me the day before going on vacations, consider it not approved.}  

\section*{Progress Reports}
At the end of each semester, graduate students are required to write a progress report for the Chemistry Department. This is very important and should be taken seriously. The more your report takes the form of a manuscript in progress, the more useful it will be for you when it is time to write the actual paper. These reports should include, at a minimum, the motivation, results, and future directions of your current project. This includes citing important literature (especially recent work). While these reports themselves will not be published, they provide a record of your work up to that point and help in organizing your data in a meaningful way (high quality images, tabulated data, etc.). It can be very hard to understand raw data in a spreadsheet (although we will want to see the raw data eventually prior to publication), processed data is much easier to think about and provide feedback on.\\

\section*{Group Meetings}
Weekly group meetings, scheduled each semester based on the group member's courses and teaching responsibilities, are the perfect opportunity for students and postdocs to develop their communication skills and get feedback on their research. Group meetings are shared with the Vlaisavljevich group. The meeting format starts with group business, then a group member will give a talk about their work telling as complete a story as possible. While the research may not be complete, the slides should be polished and carefully checked for mistakes prior to presenting. \textbf{You are expected to give a 40 min presentation.}\\

All group members should ask questions at group meeting. While it can sometimes feel like you are picking on your lab mate if you ask a tough question, remember that group meeting is the best place to get asked something hard (it is the lowest stakes presentation that you have). The better the presentation is prepared, the more meaningful the audience questions will be.  

\section*{Subgroup Meetings}
Every day, a different subgroup meeting is scheduled at 9AM (e.g., polyoxovanadate projects, actinide projects, biofilms, writing update, literature update,...).You should be on time. Different group members are expected to attend each subgroup meetings. You might be asked to attend only one, two, or rarely all. This meeting is informal but a chance for you to update me on what you have done that week. It is your chance to talk to me besides scheduling a meeting. Come to this meeting prepared to show me what you have been doing. It does not need to be a formal report, but if you are stuck on a research problem, you should prepare your results in a way that most clearly shows the problem. \textbf{Do not come empty handed.} If you have no questions, the meeting may last 5 minutes (a simple update). If you have a lot of things you want to talk about, it can go as long as time permits.

\section*{Literature}
Each group member must post the link to least 2 papers they read that week into the Literature channel in the Miro Group MS Teams. Include the title, authors, abstract, DOI, and link to the journal. \textbf{Do not include the pdf of the paper.} 

\section*{MS Teams}
All research documents besides the calculations should be in MS teams. \textbf{All documents have to be edited live through MS Office 365 and MS Teams.} Do not download files for offline editing. Copies of all your presentations, posters, reports should be in MS teams. Avoid editing previous presentations. Once you give a presentation you should keep an unedited copy of that presentation properly labeled and make a copy if you want to further edit it or expand it for a future presentation.

\section*{Reports}
Every month each group member will prepare a report that will be a compilation of results in a comprehensive form (e.g. paper, chapter of thesis, ...). If you have multiple projects it will be the more advanced one. The reports should be high quality and other group members should be able to read them and understand your project, so it requires introductions, computational details, good figures and tables, references... Reports will be live MS Word documents in teams and each report will be accompanied by an EndNote library. We will exchange reports and each group member will read another group member's report and give feed back. Everyone in Miro Group should have access to channel in this Teams group called Reports where there is a folder with the live MS Word document. Everyone in the group will do the reports including undergraduate students and postdocs. 

\subsection*{Collaborations}
Collaborations are a vital part of scientific research and we work closely with experimentalists at USD and at other institutions. When you agree to engage in a collaborative project, you are committing to providing the collaborator with the highest possible quality of data. The group policy is that with each set of data you provide, you must include the computational details (journal style). While we may get annoyed if you send us a table of data with typos in it, it is very likely that we will catch an obvious error. You cannot expect an experimental collaborator to identify when a particular method fails or when your calculation does not compare well with their measurement. It is the burden of the computational chemist to ensure that the right calculations is performed for comparison with a particular experiment. Additionally, errors in your data could lead an experimental group to perform a costly or time consuming experiment for no reason. When you first start collaborating with experimental groups, any reports or data that is to be sent to a collaborator must be approved by your advisor. We will work with you as you learn best practices in communicating your results. As you gain more experience (or if you are a postdoc), you will become responsible for managing these collaborations yourself and that includes knowing when to ask if you are not sure about a result. If you do not conform to the above criteria, you will be removed from the collaborative project. This will be particularly problematic if your funding is related to this project. If you have a question about when you can freely send data, and when you should get approval, always ask.

\end{document}
